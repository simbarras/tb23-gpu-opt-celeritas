\chapter{Generative AI and Tools}
\label{ch:tools}

To avoid all misunderstandings with the new chart of the HEIA-FR concerning the
usage of tools issued from generative AI, I want to clarify.

As these are very powerful tools, I'm totally against the idea of banning them,
and I'm looking instead for a way to increase my productivity by using them.
However, I do agree that productivity gains must not take precedence over the
right to personal property, but I do have some reservations about the HEIA-FR
guidelines.

When I code or write reports, I am accompanied by the "GitHub Copilot" tool.
This tool constantly generates suggestions that I accept, inspire or ignore.
You'll understand that it's not possible for me to define the exact lines
generated.
In addition, this raises a question: if Copilot suggests the exact line I wanted
to write and I accept its proposal, is it generated or not?
That's why you'll never find a mention that says that a line is generated unless
I find the information relevant.

I use forums like Stackoverflow or chatbots like chatGPT in two ways.
The most common is to solve a bug where I can't find the solution right away.
The second is when I don't know how to do it and need an example.
In all cases, and whatever the tool, I try to describe my context as best I can
and explain my problem.
The answers I receive are tested and understood so that they can be perfectly
adapted to my situation and integrated.
Here again, you won't find any explicit mention in the code, except in relevant
cases, as it's not possible for me to trace all my inspirations.

I should also mention that I do most of my work in English, so I use Deepl to
translate texts, sentences and words.
For proofreading, I have a premium subscription to Antidote, which improves
quality by highlighting errors.
To create the graphs, I principally use matplotlib with Python to generate the
graphs that show data.
The schemas that look hand-drawn are made with excalidraw.
Other proofreaders, translators and other tools will probably be used under my
supervision.
Other sources such as Wikipedia or other sites and articles will be read for
inspiration and mentioned if necessary.

Finally, even if there's no annotation mentioning the use of a tool, consider
that everything I've done could have been manipulated by an AI or any other tool.
However, I remain the only captain of the ship and I assure you that I
understand and am able to explain what is being produced.
This includes assuming all responsibility for the work I provide.

\chapter{Declaration of Honor}
\label{ch:honour}
I, the undersigned, Simon Barras, declare on my honor that the work rendered is the result of
personal work. I certify that I have not resorted to plagiarism or any other form of fraud.
All sources of information used and author citations have been clearly mentioned.



\chapter{Acknowledgements}
\label{ch:remerciement}

This bachelor thesis has been made by Simon Barras, a student of the \acrlong{heia}.
This includes that it is subject to the HES-SO regulations and if it achieves, it
will graduate the student.
It has been supervised by Prof. Frédéric Bapst and Prof. Jean Hennebert, both
are teachers at the institute iCoSys from the \acrshort{heia} and the expert was Dr. Baptiste Wicht.

This thesis has been made in the \acrlong{lbl} in Berkeley, CA, USA.
Simon Barras was in the team of Paolo Calafiura and he works closely with Julien
Esseiva.
The project was part of the Celeritas project which has its own acknowledgements.

\section{Celeritas}
\label{ch:acknowledgements:celeritas}

This material is based upon work supported by the U.S. Department of Energy,
Office of Science, Office of Advanced Scientific Computing Research and Office
of High Energy Physics, Scientific Discovery through Advanced Computing (SciDAC)
program.

This research was supported by the Exascale Computing Project (17-SC-20-SC), a
joint project of the U.S. Department of Energy's Office of Science and National
Nuclear Security Administration, responsible for delivering a capable exascale
ecosystem, including software, applications, and hardware technology, to support
the nation's exascale computing imperative.

This research used resources of the Oak Ridge Leadership Computing Facility,
which is a DOE Office of Science User Facility supported under Contract
DE-AC05-00OR22725.

\section{Thanks}
\label{ch:acknowledgements:thanks}

I would thank all the people that help me to do this project.
First, I would thank my supervisors which whom we have been in contact every week.
They helped me to keep focus on the objectives and they gave me some inputs to
improve my work.
I would thank Dr. Baptiste Wicht who was my expert and they have kept an eye
on my work.
His experience in the field of the \acrshort{gpu} helped me to have a better
understanding of how the \acrshort{gpu} works and how to improve the performance
of Cleritas.
I would thank Paolo Calafiura that includes me in his team and that gave me the
opportunity to work in the \acrshort{lbl}.
He was always available to answer my questions and to help me to find my places
inside the team.
I have to say a big thank you to Julien Esseiva where the person with that I work
the most with.
I asked a lot of questions and he always answered me with a lot of details.
To conclude, I would thank all the other people that helped me to do these
experiments like my family for their support but also the staff at the \acrshort{heia}
that helped me to get my visa.

\chapter{Software Version}
\label{ch:software}
These are all the tools with their version used in this project.

\begin{table}[ht]
    \centering
    \begin{tabular}{|l|p{3cm}|l|}
    \hline
    \multicolumn{1}{|c|}{\textbf{Software}} & \multicolumn{1}{c|}{\textbf{Version}} & \multicolumn{1}{c|}{\textbf{Description}} \\ \hline
    xxx                     & v0.0                                  & -        \\ \hline
    \end{tabular}
    \caption{Software version}
    \label{tab:softwareVersion}
\end{table}