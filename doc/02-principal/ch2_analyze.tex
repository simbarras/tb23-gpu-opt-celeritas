\chapter{Analysis}
\label{ch:analyze}

During this chapter, the project is analyzed to understand the context, the
goals and challenges.
This project is focused on the optimization of the particle's tracking using
\acrshort{gpu}s.
\acrshort{gpu}'s programming is something totally new for a bachelor's
student at the \acrshort{heia}.
Celeritas is a project that is actively developed and it requires time to
understand the code and the architecture.


\section{GPU}
\label{ch:analyze:gpu}

The \acrfull{gpu} is a processor that is specialized in parallel computing.
It is known by the gamer community because it is used to render the graphics of
the games and a good graphic card increases the frames per second displayed.
However, in the professional world, the \acrshort{gpu} are becoming more and
more popular because they are more efficient than the \acrshort{cpu} for
parallel computing and managing a large amount of data.


\subsection{Use cases}
\label{ch:analyze:gpu:use-cases}

In most cases, \acrshort{cpu}s are more efficient than \acrshort{gpu}s
which is why they are a mandatory component of a computer.
\acrshort{cpu}s have a high frequency and they are optimized to execute one
task at a time with a high efficiency.
On the other hand, \acrshort{gpu}s have a lower frequency and they are optimized
to execute multiple tasks at the same time.

To understand, imagine John wants to eat dinner with his 5 kids and they have to
cook something and dress the table.
Cooking is a task that must be done sequentially and require some skills and
efficiency to be done quickly.
Dressing the table is a task where the skills don't affect the time but the
number of hands does.
This task can be done by the kids when everyone is putting something on the table.
In our case, John is the \acrshort{cpu} and the kids are the \acrshort{gpu}s.

The \acrshort{gpu}s are used when the number of workers is more important than
the efficiency of the worker like computing the pixels of an image or the
particles of a simulation.




\subsection{Architecture}
\label{ch:analyze:gpu:architecture}

The table \ref{tab:analyze:gpu:architecture:instruction-data} shows the
different types of processors.
The capabilities to work with one or multiple data and instructions are defined
by the architecture of the processor.

\begin{table}[ht]
    \centering
    \begin{tabular}{l|l|l|}
                          & Single data & Multiple data \\ \hline
    Single instruction    & SISD        & \textbf{SIMD}  \\ \hline
    Multiple instructions & MISD        & MIMD           \\ \hline
    \end{tabular}
    \caption{Different types of processors working with one or multiple data and instructions}
    \label{tab:analyze:gpu:architecture:instruction-data}
\end{table}

The \acrshort{gpu} is a \acrshort{simd} processor and the \acrshort{cpu} is a
\acrshort{sisd} processor.
The table \ref{tab:analyze:gpu:architecture:instruction-data} does not show the
frequency of the processor is an important factor for performance.
This explains why the \acrshort{cpu} is more efficient than the \acrshort{gpu}
for most of the cases.

To deal with the \acrshort{simd} architecture, the \acrshort{gpu} executes
physically the same instruction on multiple data using multiple threads as shown
on Figure \ref{fig:analyze:gpu:architecture:smid}.

\image{0.35}{05-resources/img/analyze/simd.excalidraw.png}
{SIMD physically executed}
{fig:analyze:gpu:architecture:smid}

The \acrshort{gpu} is composed of multiple \acrfull{sm} that are composed of
multiple \acrfull{cuda} cores.
The \acrshort{cuda} cores are the processors that execute the instructions.
The \acrshort{sm} is the unit that manages the threads and the memory.
Figure \ref{fig:analyze:gpu:architecture:sm} shows the architecture of a \acrshort{sm}.

\image{0.65}{05-resources/img/analyze/sm.png}
{\acrshort{sm} architecture~\cite{nvidia-a100-architecture}}
{fig:analyze:gpu:architecture:sm}

Figure \ref{fig:analyze:gpu:architecture:sm} introduces the warp that is a
group of 32 threads that are executed in parallel.
The threads of a warp are linked together and they have to wait for the others
to finish their instructions.


\section{CUDA}
\label{ch:analyze:cuda}

\acrfull{cuda} is a parallel computing platform and programming model developed
by Nvidia.
It allows developers to use the \acrshort{gpu} to execute code written in C++, C, Fortran
and Python.
The \acrshort{cuda} platform comes with a \acrshort{sdk} that provides
libraries, debugging and profiling tools.

To develop with \acrshort{cuda}, the \acrshort{sdk} must be installed and all
the examples in this report are with C++ because it is the language used in the
project.

It is recommended to have some basic knowledge to understand in \acrshort{gpu}
programming because the following sections are not written as a tutorial but as
a reminder to have all the most important information in one place.
The course~\cite{cuda-training} which has been dispensed at \acrshort{ornl} is
a good way to learn \acrshort{cuda}.

\subsection{Basis}
\label{ch:analyze:cuda:basis}

The code that runs on a \acrshort{gpu} is in a function called "kernel" and it
is executed by every thread.
Those kernels are defined with the keyword \texttt{\_\_global\_\_} and they are called
with the function \texttt{kernel\_name<<<number\_of\_blocks, number\_of\_threads>>>()}.
The code \ref{code:analyze:cuda:develop:basis:kernel} shows a basic kernel
launch with 2 blocks of 3 threads which means that the kernel code will be
executed 6 times.

\begin{code}
    \captionof{listing}{Basic kernel example}
    \label{code:analyze:cuda:basis:kernel}
    \begin{minted}{C++}
__global__ void kernel_name() {
    // Code executed on the device
}

int main() {
    // Code executed on the host
    kernel_name<<<2, 3>>>();
    return 0;
}
    \end{minted}
\end{code}

\subsection{Memory}
\label{ch:analyze:cuda:memory}

As the \acrshort{gpu} is a separate processor, it has its own memory system.
As a developer we have to manage three different spaces where data can be stored:
\begin{itemize}
    \item Host memory (RAM)
    \item Device memory (DRAM)
    \item Shared memory
\end{itemize}

Every space becomes with its properties and using the right one is important to
have a good performance.
Figure \ref{fig:analyze:cuda:memory:architecture} shows the physical
organization of the memory on a \acrshort{gpu}.

\image{1}{05-resources/img/analyze/gpu-memory.png}
{Physical memory organization on a \acrshort{gpu}~\cite{cuda-training}}
{fig:analyze:cuda:memory:architecture}


\subsubsection{Host memory}
\label{ch:analyze:cuda:memory:host}

Memory allocated on the host is the memory we use every day but it is not
accessible by the \acrshort{gpu}.
Even if a program is using the \acrshort{gpu}, it still needs to use the host
memory.

\subsubsection{Device memory}
\label{ch:analyze:cuda:memory:device}

The device memory is the memory that is used by the \acrshort{gpu} to execute
the kernel and it is comparable to a heap and a stack memory.
This place contains the local variables of a thread, the global variables and
the constant memory.
This memory is accessible by all the threads of a grid and the data can be
transferred from/to the host memory via specific statements on the host code.

The arguments passed to a kernel can only be 64 bits long so the object must be
passed by reference and the value must be copied in the device memory.
To get the result, the data must be copied back to the host memory.
The code \ref{code:analyze:cuda:memory:device:copy-read} shows how to launch a
kernel with objects and integers as arguments.
As the kernel isn't executed on the host, the return keyword can't be used to
get the result of the kernel so the result must be copied back to the host
using the same mechanism as the arguments.

\begin{code}
    \captionof{listing}{Load ang read data from the device memory}
    \label{code:analyze:cuda:memory:device:copy-read}
    \begin{minted}{C++}
__global__ void kernel_name(ObjectType1 *input1,
                            int input2,
                            ObjectType2 *output) {
    ObjectType3 local_variable;
    output[threadIdx.x] = ObjectType2(input1[threadIdx.x] + input2);
}

int main() {
    // Instantiate the variables
    ObjectType1 *h_input1, *d_input1;
    ObjectType2 *h_output, *d_output;

    // Setting the values
    h_input1 = new ObjectType1[100];
    for (int i = 0; i < 100; i++) {
        h_input1[i] = ObjectType1(i);
    }
    h_output = new ObjectType2[100];
    int input2 = 5;

    // Allocate the memory on the device
    cudaMalloc(&d_input1, 100 * sizeof(ObjectType1));
    cudaMalloc(&d_output, 100 * sizeof(ObjectType2));

    // Copy the data from the host to the device
    cudaMemcpy(d_input1,
               h_input1,
               100 * sizeof(ObjectType1),
               cudaMemcpyHostToDevice);

    // Launch the kernel
    kernel_name<<<1, 100>>>(d_input1, input2, d_output);

    // Copy the data from the device to the host
    cudaMemcpy(h_output,
               d_output,
               100 * sizeof(ObjectType2),
               cudaMemcpyDeviceToHost);

    // Free the memory
    cudaFree(d_input1);
    cudaFree(d_output);

    // Do something with the output
    print(h_output);

    return 0;
}
    \end{minted}
\end{code}

Constant expressions and methods that will be used on the device must have the
keyword \texttt{\_\_device\_\_}.
To keep a copy on the host, the double keyword
\texttt{\_\_host\_\_ \_\_device\_\_} muste be used.


\subsubsection{Shared memory}
\label{ch:analyze:cuda:memory:shared}

The shared memory is a memory that is shared between the threads of a block.
It is quicker than the device memory but it is limited to 48~KB per block.
The shared memory is used to share data between the threads of a block and to
cache data from the global memory or exchange data between the threads.

The shared memory is allocated with the keyword \texttt{\_\_shared\_\_} and it
can be done statically (see code~\ref{code:analyze:cuda:memory:shared:static}) or
(see code dynamically~\ref{code:analyze:cuda:memory:shared:dynamic}).
As the examples are very close to the code \ref{code:analyze:cuda:memory:device:copy-read},
the details to copy the data from the host to the device and from the device to
the host are not shown.

\begin{code}
    \captionof{listing}{Static shared memory allocation}
    \label{code:analyze:cuda:memory:shared:static}
    \begin{minted}{C++}
__global__ void kernel_name(ObjectType1 *input,
                            int input2,
                            ObjectType2 *output) {
    __shared__ ObjectType3 shared_variable[100];
    shared_variable[threadIdx.x] = ObjectType3(input[threadIdx.x] + input2);
    int index_next_thread = (threadIdx.x + 1) % 100;
    __syncthreads();
    output[threadIdx.x] = ObjectType2(shared_variable[index_next_thread]);
}

int main() {
    // Instantiate the variables
    // Setting the values
    // Allocate the memory on the device
    // Copy the data from the host to the device
    // Launch the kernel
    kernel_name<<<1, 100>>>(d_input1, input2, d_output);

    // Copy the data from the device to the host
    // Free the memory
    // Do something with the output
    return 0;
}
    \end{minted}
\end{code}

\begin{code}
    \captionof{listing}{Dynamic shared memory allocation}
    \label{code:analyze:cuda:memory:shared:dynamic}
    \begin{minted}{C++}
__global__ void kernel_name(ObjectType1 *input,
                            int input2,
                            ObjectType2 *output) {
    extern __shared__ ObjectType3 shared_variable[];
    shared_variable[threadIdx.x] = ObjectType3(input[threadIdx.x] + input2);
    int index_next_thread = (threadIdx.x + 1) % 100;
    __syncthreads();
    output[threadIdx.x] = ObjectType2(shared_variable[index_next_thread]);
}

int main() {
    // Instantiate the variables
    // Setting the values
    // Allocate the memory on the device
    // Copy the data from the host to the device
    // Compute the size of the shared memory
    int size_shared_memory = 100 * sizeof(ObjectType3);

    // Launch the kernel
    kernel_name<<<1, 100, size_shared_memory>>>(d_input1, input2, d_output);

    // Copy the data from the device to the host
    // Free the memory
    // Do something with the output
    return 0;
}
    \end{minted}
\end{code}


\subsection{Synchronization}
\label{ch:analyze:cuda:synchronization}

The easiest way to use \acrshort{gpu} is to have a set of data and using one
thread to update one data, for instance when you want to compute the sum of two
vectors.
Next, there are multiple dimension problems with matrix or 3D matrix addition.
The most difficult type of problem is the reduction where the threads must
communicate between them to get the result. For example, the sum of all the
elements of a vector is illustrated in Figure \ref{fig:analyze:cuda:synchronization:reduction}.

\image{0.5}{05-resources/img/analyze/reduction-problem.png}
{Reduction problem~\cite{cuda-training}}
{fig:analyze:cuda:synchronization:reduction}

\acrshort{cuda} provides some low-level function to synchronize, preserve the
integrity or exchange data.

\subsubsection{Atomic operations}
\label{ch:analyze:cuda:synchronization:atomic}

Atomic operations are used to preserve the integrity of the data when multiple
threads are trying to access the same data.
For example, if two threads are trying to increment a shared integer, the
result will be wrong because the two threads will read the same value and write
the same value so the result will be incremented only once.

Every atomic operation requires a pointer to the data that could eventually be
modified and a value.
The value could have different roles but the instruction is returning the old
value stored there.
The different atomic operations are listed in Table \ref{tab:analyze:cuda:synchronization:atomic}.

\begin{table}[ht]
    \centering
    \begin{tabular}{|m{0.5\textwidth}|m{0.4\textwidth}|}
        \hline
        \textbf{Operation} & \textbf{Description} \\
        \hline
        \texttt{atomicAdd/Sub(addr, val)} & Add a value to an integer \\
        \hline
        \texttt{atomicMin/Max(addr, val)} & Set the minimum/maximum value \\
        \hline
        \texttt{atomicInc/Dec(addr, val)} & Increment/Decrement an integer if the new value will be from 0 to val\\
        \hline
        \texttt{atomicCAS(addr, compare, val)} & Swap value to addr if old is equal to compare\\
        \hline
        \texttt{atomicExch(addr, val)} & Swap value to addr \\
        \hline
        \texttt{atomicAnd/Or/Xor(addr, val)} & Bitwise And/Or/Xor \\
        \hline
    \end{tabular}
    \captionof{table}{CUDA atomic operations}
    \label{tab:analyze:cuda:synchronization:atomic}
\end{table}

\subsubsection{Warp shuffle}
\label{ch:analyze:cuda:synchronization:warp-shuffle}

The warp shuffle is a feature that allows developers to exchange data between the threads
of a warp.
This mechanism is limiting the time waste to exchange data using shared memory
but it is only working for 4 or 8 bytes.

Those warp instructions are listed in Table \ref{tab:analyze:cuda:synchronization:warp-shuffle}.
They return the value of the variable specified from the thread lane specified.
The mask is a 32-bit unsigned int that represents the lane-id in one-enabled bit.
Only the threads specified by this mask will wait for the synchronization.
Var is the name of the variable that will be pulled from the thread lane.

\begin{table}[ht]
    \centering
    \begin{tabular}{|m{0.5\textwidth}|m{0.4\textwidth}|}
        \hline
        \textbf{Operation} & \textbf{Description} \\
        \hline
        \texttt{\_\_shfl\_sync(mask, var, lane)} & Get the value from the thread lane\\
        \hline
        \texttt{\_\_shfl\_up/down\_sync(mask, var, delta)} & Get the value from the thread lane $\pm$ delta\\
        \hline
        \texttt{\_\_shfl\_xor\_sync(mask, var, laneMask)} & Get the value from the thread lane \^{} laneMask\\
        \hline
    \end{tabular}
    \captionof{table}{CUDA warp shuffle operations}
    \label{tab:analyze:cuda:synchronization:warp-shuffle}
\end{table}


\subsubsection{Sync}
\label{ch:analyze:cuda:synchronization:sync}

Warp shuffle offers a synchronization method but \acrshort{cuda} provides
functions to synchronize the threads.
This could be useful if the threads have to exchange data between them using
the shared memory.

Table \ref{tab:analyze:cuda:synchronization:sync} shows the different
synchronization functions.

\begin{table}[ht]
    \centering
    \begin{tabular}{|m{0.5\textwidth}|m{0.4\textwidth}|}
        \hline
        \textbf{Operation} & \textbf{Description} \\
        \hline
        \texttt{\_\_syncthreads()} & Synchronize all the threads of a block\\
        \hline
        \texttt{\_\_threadfence()} & Synchronize all memory accesses of a block\\
        \hline
        \texttt{\_\_synchwarp(mask)} & Synchronize the threads of a warp that match the mask (default: all)\\
        \hline
    \end{tabular}
    \captionof{table}{CUDA synchronization functions}
    \label{tab:analyze:cuda:synchronization:sync}
\end{table}


\subsection{Create a project with CUDA}
\label{ch:analyze:cuda:create_project}

\acrshort{cuda} programming can be easily integrated in any C++ project.
The main constraint is to have a \acrshort{gpu} to execute the code and it is
easy to build on it to be sure that version is the same.
If the working station doesn't have a \acrshort{gpu}, it is possible to use a
SSH connection to a server with a \acrshort{gpu}.

To do that with CLion, the first step is to have a project and then in the
settings, add a toolchain with the \acrshort{cuda} compiler~\ref{fig:analyze:cuda:toolchain}.
Then in the CMake profile set the toolchain set before~\ref{fig:analyze:cuda:profile}.

Usually, the kernels are defined in a \texttt{.cu} file and the host code is in
a \texttt{.cc} file but everything can be in the same file.
To run the code, the most convenient way is to use CMake to compile the code
and then run the executable.
Code \ref{code:analyze:cuda:create_project} shows a basic CMake file to
compile a project with \acrshort{cuda}.

\begin{code}
    \captionof{listing}{Basic CMake file to compile a project with CUDA}
    \label{code:analyze:cuda:create_project}
    \begin{minted}{CMake}
cmake_minimum_required(VERSION 3.16)
project(project_name LANGUAGES CXX CUDA)

set(CMAKE_CXX_STANDARD 14)

# add CUDA
find_package(CUDA REQUIRED)

# add executable
add_executable(executable_name executable_file.cu)
    \end{minted}
\end{code}

\image{0.8}{05-resources/img/analyze/clion-toolchain.png}
{Add CUDA toolchain}
{fig:analyze:cuda:toolchain}

\image{0.8}{05-resources/img/analyze/cmake-profile.png}
{Set profile to toolchain}
{fig:analyze:cuda:profile}



\section{Celeritas}
\label{ch:analyze:atlas}

Celeritas is a \acrfull{hep} detector simulation on \acrshort{gpu}s started
early 2020.
This project purposed to the large amount of data produced by the
\acrfull{lhc} that is not possible to be processed by traditional
\acrshort{cpu}s.
The project is actively developed by five laboratories that are: \acrfull{lbl},
\acrfull{ornl}, \acrfull{anl}, \acrfull{fnal} and \acrfull{bnl}.
The code leader is Seth Johnson from \acrshort{ornl} and Julien Esseiva is the only
participant from the \acrshort{lbl}.

\image{0.75}{05-resources/img/analyze/celeritas-organization.png}
{Celeritas actors and roles~\cite{celeritas-presentation-johnson}}
{fig:analyze:atlas:organization}

The goal is to simulate electromagnetic physics for the \acrshort{lhc} detectors
with the same precision.
The application could act as a plugin for Geant4, or a standalone application.


\subsection{Geant4}
\label{ch:analyze:atlas:geant4}

Celeritas has an end-to-end solution that can be launched standalone.
This version allows the developer to have full control of the simulation
workflow and choose the geometry motors and physics models.
However, the standalone version needs a lot of development efforts to be as
complete as Geant4.

\image{1}{05-resources/img/analyze/celeritas-end2end-integration.png}
{Integration diagram of the standalone application~\cite{celeritas-overview-tognini}}
{fig:analyze:atlas:geant4:end2end}


Celeritas has a version called "Acceleritas" that has the same code base as the
standalone version but is used as a Geant4 plugin.
This version provides less improvement but all the Geant4's features are available.

\image{1}{05-resources/img/analyze/celeritas-acceleritas-integration.png}
    {Integration diagram for overloading of Geant4~\cite{celeritas-overview-tognini}}
    {fig:analyze:atlas:geant4:plugin}

Geant4~\cite{geant4} is a toolkit for the simulate the path of particles through
matter.
It is used in multiple fields, including the experiments made in the
accelerator physics.
This tool is developed by the \acrfull{cern} and is currently used by the
\acrfull{atlas} and \acrfull{cms} experiments.


\subsection{Runge Kutta Dormand Prince}
\label{ch:analyze:atlas:rkdp}

The \acrfull{rkdp} is a numerical method to solve ordinary differential equations.
It is defined via the coefficients from the Butcher tableau which is a low triangular
matrix.
The method is used to simulate the path of a particle through matter and
magnetic fields to a boundary of a sensor.

This is the Butcher tableau~\ref{tab:analyze:atlas:rkdp:butcher} for the \acrshort{rkdp} used in Celeritas.

\begin{table}[ht]
    \resizebox{\textwidth}{!}{
        \begin{tabular}{c|ccccccc}
            0    &            &             &            &          &               &          &      \\
            1/5  & 1/5        &             &            &          &               &          &      \\
            3/10 & 3/40       & 9/40        &            &          &               &          &      \\
            4/5  & 44/45      & -56/15      & 32/9       &          &               &          &      \\
            8/9  & 19372/6561 & -25360/2187 & 64448/6561 & -212/729 &               &          &      \\
            1    & 9017/3168  & -355/33     & 46732/5247 & 49/176   & -5103/18656   &          &      \\
            1    & 35/384     & 0           & 500/1113   & 125/192  & -2187/6784    & 11/84    &      \\
            \hline
                 & 35/384     & 0           & 500/1113   & 125/192  & -2187/6784    & 11/84    & 0    \\
                 & 5179/57600 & 0           & 7571/16695 & 393/640  & -92097/339200 & 187/2100 & 1/40 \\
        \end{tabular}
    }
    \captionof{table}{Butcher tableau for the \acrshort{rkdp}}
    \label{tab:analyze:atlas:rkdp:butcher}
\end{table}



\subsection{Particle's path}
\label{ch:analyze:atlas:path}

Celeritas simulates the path of a particle through matter and magnetic fields.
The path of a particle is calculated by the \acrshort{rkdp}~\ref{ch:analyze:atlas:rkdp} method.

\todo{add an image of a particle's path}

As Celeritas is actively developed, not all the particles are now implemented.
The team focused on the High Luminance experiments and the particles that are
implemented are listed in Table \ref{tab:analyze:atlas:particles:implemented}.

\begin{table}[ht]
    \centering
    \begin{tabular}{lll}
        \hline
        \textbf{Particle}         & \textbf{Process}     & \textbf{Model(s)}            \\
        \hline
        \multirow{4}{*}{$\gamma$} & photon conversion    & Bethe—Heitler                \\
                                  & Compton scattering   & Klein—Nishina                \\
                                  & photoelectric effect & Livermore                    \\
                                  & Rayleigh scattering  & Livermore                    \\
        \hline
        \multirow{4}{*}{$e^\pm$}  & ionization           & Mø11er-Bhabha                \\
                                  & bremsstrahlung       & Seltzer—Berger, relativistic \\
                                  & pair annihilation    & EPlusGG                      \\
                                  & multiple scattering  & Urban, WentzelV1             \\
        \hline
        $\mu^\pm$                 & muon                 & Muon Bremsstrahlung          \\
        \hline
    \end{tabular}
    \captionof{table}{Particles now implemented in Celeritas~\cite{exasclae-computing-ornl-evans}}
    \label{tab:analyze:atlas:particles:implemented}
\end{table}

The team is working on expanding the project to other experiments and has planned
to add the following particles~\ref{tab:analyze:atlas:particles:planned}.

\begin{table}[ht]
    \centering
    \begin{tabular}{lll}
        \hline
        \textbf{Physics}          & \textbf{Process}       & \textbf{Particle(s)}              \\
        \hline
        \multirow{10}{*}{EM}      & Photon conversion      & $\gamma$                          \\
                                & pair annihilation      & $e^\pm$                           \\
                                & photoelectric effect   & $\gamma$                          \\
                                & ionization             & charged leptons, hadrons and ions \\
                                & bremsstrahlung         & charged leptons and hadrons       \\
                                & Rayleigh scattering    & $\gamma$                          \\
                                & Compton scattering     & $\gamma$                          \\
                                & Coulomb scattering     & charged leptons and hadrons       \\
                                & multiple scattering    & charged leptons and hadrons       \\
                                & continuous energy loss & charged leptons, hadrons and ions \\
        \hline
        \multirow{3}{*}{Decay}    & two body decay         & $\mu^\pm$, $\tau^\pm$, hadrons    \\
                                & three body decay       & $\mu^\pm$, $\tau^\pm$, hadrons    \\
                                & n-body decay           & $\mu^\pm$, $\tau^\pm$, hadrons    \\
        \hline
        \multirow{6}{*}{Hadronic} & photon-nucleus         & $\gamma$                          \\
                                & lepto-nucleus          & leptons                           \\
                                & nucleon-nucleon        & $p$, $n$                          \\
                                & hadron-nucleon         & hadrons                           \\
                                & hadron-nucleus         & hadrons                           \\
                                & nucleus-nucleus        & hadrons                           \\
        \hline
    \end{tabular}
    \caption{Particles planned in the future version of Celeritas~\cite{exasclae-computing-ornl-evans}}
    \label{tab:analyze:atlas:particles:planned}
\end{table}


\subsection{Implementation}
\label{ch:analyze:atlas:implementation}

Celeritas is realized in C++ and uses CUDA, HIP or OpenMP to accelerate the
simulation.
The code is data-oriented and is using composition instead of inheritance, and
to optimize the kernel launching on the \acrshort{gpu}, all the data are loaded
one time at the beginning of the simulation.

Celeritas is using a flow of action to process the data given, compute the
tracks, simulate the collisions and finally return the results as shown in the
figure \ref{fig:analyze:atlas:implementation:activity-diagram-gpu-topological}.

\image{0.75}{05-resources/img/analyze/celeritas-GPU-loop-topological.png}
{Celeritas's actions activity diagram~\cite{chep2023-presentation-johnson}}
{fig:analyze:atlas:implementation:activity-diagram-gpu-topological}

My own Bachelor project is focusing on optimizing a part of particle tracking which is
made in the action called "Along Step".
This action is executed for each particle by one dedicated \acrshort{gpu}'s
thread.
Figure \ref{fig:analyze:atlas:implementation:activity-diagram-track} shows
the activity diagram of the particle's track.

The method \acrshort{rkdp} (see Section~\ref{ch:analyze:atlas:rkdp}) is used to calculate the
chord of the path during the action "Calculate substep from remaining distance"
in Figure \ref{fig:analyze:atlas:implementation:activity-diagram-track}.
The chord, with different step sizes, is computed until the error meets the
tolerance or the number of tries is reached.
If the tolerance is not met, an ultimate iteration will be made and the error
is provided.

According to the result of an experiment made by the \acrfull{cms}.
Figure~\ref{fig:analyze:atlas:implementation:iteration-hit} shows our statistics
which say that in 80\% of the cases, the tolerance is met at the first
iteration.

\image{1}{05-resources/img/analyze/celeritas-iteration-hit.png}
{Number of iterations needed to meet the tolerance for each computed chord}
{fig:analyze:atlas:implementation:iteration-hit}

\image{0.5}{05-resources/img/analyze/activity-diagram-track.png}
{Activity diagram of a particle's track~\cite{atlas-week-esseiva}}
{fig:analyze:atlas:implementation:activity-diagram-track}

\subsection{Optimization}
\label{ch:analyze:atlas:optimization}

Celeritas is already accelerated by \acrshort{gpu}s, but it is suspected that
the performance can be improved.

To simulate the particles, Celeritas uses one \acrshort{gpu} thread to track one particle.
All the chords are simulated one after the other on a kernel and a chord needs
one or more \acrshort{rkdp} iterations.

The team wants to improve performance by using more than one thread per
track.

\image{0.5}{05-resources/img/analyze/celeritas-optimization.png}
{Celeritas runtime per action~\cite{chep2023-presentation-johnson}}
{fig:analyze:atlas:optimization:celeritas-optimization}





\section{Images Backup}
\label{ch:analyze:images}

\image{0.5}{05-resources/img/analyze/celeritas-GPU-loop-branch.png}
{Celeritas activity diagram for the GPU loop (Source: Seth Johnson https://github.com/celeritas-project/celeritas-docs/blob/main/presentations/chep-2023/srj-chep.pdf)}
{fig:analyze:atlas:implementation:activity-diagram-gpu}

\image{0.5}{05-resources/img/analyze/field-propagator-near-miss.png}
{Path simulation with chord (Source: Set Johnson https://github.com/celeritas-project/celeritas-docs/blob/main/presentations/doe-briefing-20201019/pres.pdf)}
{fig:analyze:atlas:implementation:field-propagator-near-miss}

\image{0.5}{05-resources/img/analyze/celeritas-kernel.png}
{Detail of the kernels (Source Seth Johnson https://github.com/celeritas-project/celeritas-docs/blob/main/presentations/geant4-collab-202209/Celeritas.pdf)}
{fig:analyze:atlas:implementation:kernel}

\image{0.5}{05-resources/img/analyze/celeritas-gpu-algorithm.png}
{GPU algorithm (Source: Seth Johnson https://github.com/celeritas-project/celeritas-docs/blob/main/presentations/lbnl-202302/srj-lbnl-2023.pdf)}
{fig:analyze:atlas:implementation:gpu-algorithm}
