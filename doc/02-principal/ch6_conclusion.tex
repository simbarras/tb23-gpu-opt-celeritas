\chapter{Conclusion}
\label{ch:conclusion}

This chapter is concluding the thesis, my stay at the \acrshort{lbl} and my
studies at the \acrshort{heia}.
The two first conclusions are objective and concerning the project and the thesis.
The third one is personal and concerns my experiments during the three last
years with a focus on the last 3 months.


\section{Thesis conclusion}
\label{ch:conclusion:thesis}

This chapter compares the goals defined Section~\ref{ch:introduction:objectives}.

First of all, the goal to learn \acrshort{gpu} programming has been achieved.
The chapter about that in the documentation was not written as a tutorial but
as a reminder for the people that know how to deal with it.
The knowledge gained during this goal has been used during the whole project
and it has been improved too.

Then, to improve the project, a dive into what has been necessary.
However, with a bit of distance, I think too much time has been spent here and
the gained knowledge or facilities has not been used during the third objective.

The study of \acrshort{rkdp} has helped a lot to do two implementations of a
distributed \acrshort{rkdp}.
The two implements have been demonstrated that they are not improving the times for
a large amount of data but I think this project has been very interesting and
some things can be kept.

\section{Project conclusion}
\label{ch:conclusion:project}

Even if the main goal was to do a Bachelor's thesis to validate the Bachelor of
Simon Barras, the project was also a need for the Celeritas project.

The implementation made to improve the runtime of the \acrshort{rkdp} have some
really good performance for a small number of particles.
These implementations are not useful at the point that Celeritas needs to simulate
a large number of particles.
The reason for that is the \acrshort{gpu}s have not enough \acrshort{sm} to run
all the threads in parallel.
As using four times more threads per track, the threads have to wait to have a
\acrshort{sm} available to run.

The only way to be able to use these implementations is to have a \acrshort{gpu}
with more \acrshort{cuda} cores.
One of the possible solutions is to wait for the next generation of \acrshort{gpu} or
to find a way to distribute the vector multiplication with fewer threads.

To resume, the implementations are a Ferrari and a Pagani that we want to use to
do some off-road.

\section{Personal conclusion}
\label{ch:conclusion:personal}

This project was a great opportunity for me to learn a lot of things.

I discover a new country and I practice my English.
It was the first time that I was so far from my home for more than one month.
The fact that I was alone was a challenge for me, but I found a new way to enjoy
my life.
This experience forces me too to be more independent and to deal with a very
difficult administrative system that is the American one.

Concerning life in the \acrshort{lbl}, I was very lucky to be in a team with
people from all over the world.
The coffee time was a great moment to share our culture and learn more about
the other.
This place is very special too because it is situated near the Silicon Valley
and there are a lot of people with a high level of education.
The events organized in place are very interesting to discover or learn more
about a lot of subjects.
I have seen that Machine Learning and Quantum Computing are very
popular.
I would really enjoy coming back to California to work after my career.

About my project, I enjoy discovering a new way of programming.
I really love to deal with the constraint and the advantages of the \acrshort{gpu}.
I personally think that the experiments done will help the Celeritas project to
have a better understanding of how \acrshort{rkdp} could be improved.
I found that my project organization was good and even with the distance, I was
able to keep my supervisors and my expert informed about the progress of my work.
If I have to do it again, I will probably try to start my documentation earlier
but I think it is a good document to trace my work.

To conclude, I would say that I am very happy to be graduated (This year I hope)
at the \acrshort{heia}.
It was not easy every day, I remember my first year in German and the evaluation
of trigonometry where I understand that "Dreieck" means triangle.
All the calls that I made with Nicolas Terreaux to finish our projects and all
the coffee we drink to stay awake.
I have some really good experiences like Eurobot in spring 2022 where we finish
at the highest ranking that the school has ever done and, of course, this
amazing experience that gives me the wish to discover even more about the world and
the computer science.
To finish, I would like to thank all the people that I meet during my studies.
All the teachers, the staff and, of course, all the students and I hope that I
will have the luck to work with some of them in the future.
