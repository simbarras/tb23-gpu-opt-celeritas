\chapter{Conclusion}
\label{ch:conclusion}
% -----------------------------------------------------------------------------

Since the beginning of the project, it was clear that the project will not be finished at the end of this semester.
He was programmed to be realized by two students with the possibility of continuing this project in the bachelor thesis.

\section{Summary of the work done}
\label{ch:conclusion:summary}

The work done is above the expectations of the beginning of the project.
The uBoMa isn't totally controlled, the rotary valve is fully implemented like the readability of the pressure sensor.
With the pressure sensor, is connected to a Mikroe device that is connected to the Rapsberry Pi through the I2C protocol.
The pressure controller is not implemented du to the lack of time.
The SPI protocol wasn't studied but it's necessary to control the current loop.

The web application took more time than planned at the beginning of the project.
To this part, the project is well advanced.
The basis of the web application is done, it can interact with the core application with the REST API and the web socket.
All the devices or the actions are not implemented but it can easily be extended.

The documentation is made to help anyone who wants to continue this project.
They are some guidelines and some explanations about the technologies used.

To conclude, this semester project has been here to fix the limits of the project and find the best way to continue this project.
From this point of view, the project is a success.


\section{Problems encountered}
\label{ch:conclusion:problems}

This project has been challenging for a computer scientist that has never worked with electronics.
The choice of the devices was a difficult task because there are a lot of parameters to take into account.
During the realization, a lot of problems were encountered, like how the alimentation for the pressure sensor and controller that wasn't taken into account at the beginning of the project.
The lack of documentation for the adapters from Mikroe was a problem too.
They gave some examples but it needs an SDK which is paying.

C++ is also a language that is difficult to use and it didn't provide a lot of facilities to the programmer.
This is difficult to create a pipeline that use the cross compilation and test the project because it needs some libraries that are directly installed on the Raspberry Pi.


\section{Future work}
\label{ch:conclusion:future}

As mentioned before, this project will be continued and this semester project was here to find the limits and how to continue.

\subsection{Pressure controller}
\label{ch:conclusion:future:pressure_controller}

First of all, the pressure controller need to be implemented.
This step needs the usage of The SPI protocol.
I recommend to search a driver for the device current-loop T 2 click from Mikroe that works with the Raspberry Pi.
If it's not possible, the driver needs to be implemented.


\subsection{Web application}
\label{ch:conclusion:future:web_application}

The web application is not finished.
For each device, it could be interesting to add a card the advanced page.
This card will contain the information about the device and the possibility to change the configuration.
For the existing devices, some actions could be added.
To do that, some new endpoints need to be added to the REST API and the information sent through the web socket need to be extended.
This application can also be declined into a mobile application that will communicate with the core application through Bluetooth.


\subsection{Production environment}
\label{ch:conclusion:future:production_environment}

The production environment is not implemented.
The core application needs to implement the \acrfull{opc}.
The \acrfull{mtp} is a must-have but before it needs to be analyzed.


\subsection{Baseboard}
\label{ch:conclusion:future:baseboard}

One of the goals of the project was to reduce the size and number of wires of the whole product.
Actually, to connect the different devices there are a lot of wires and they need to be connected alimented externally because the Raspberry Pi can't provide enough power.
To sell this product to a customer, it could be a good idea to create a custom baseboard.
As the project is on Raspberry Pi 4, it can easily be ported to the Raspberry Pi Compute Module 4.
This module contains only the processor and the memory, the rest of the components need to be added on a custom baseboard.
To do that, the Raspberry Pi Foundation provides a reference design that can be used to create a custom baseboard.
It could be a good thing to add an electronic engineer to the team to help the computer scientist to create this baseboard and other tasks.


\section{Personal thoughts}
\label{ch:conclusion:personal}

This project was a good experience for me.
I learned a lot of things about electronics and how to deal with the embedded world.
This work was quite new for me and I'm happy to have done it.

I wish I could have done more things but I lost a lot of times to understand how the devices work and how to use them.
I was also new to C++ and I spend sometimes to create the best architecture for the project and the CMake files.

For me, I wish I could have done more, but I hope that what I have provided will allow Mr. Vial to continue this project.
I feel that I have provided the necessary documentation by doing the PVs every week and by bringing mini presentations despite some delays.
I also regret the time I could not spend on the project during the last weeks because of my visa problems for my bachelor work in Berkeley.
It is true that I did not necessarily keep up to date with the schedule that was set up at the beginning of the project, but I find it very difficult to anticipate the tasks of a 12-week project.
Finally, I am quite happy with the documentation provided and I hope that it can help future people who will work on this project.
