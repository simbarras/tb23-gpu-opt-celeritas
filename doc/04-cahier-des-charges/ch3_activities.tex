\chapter{Activities}
\label{spec:ch:activities}

Activities are all the tasks that need to be done to complete the project.
These tasks come directly from the goals and the planning is based on the activities listed above.

All the tasks are not mentioning the documentation, but it's implicit that all the tasks will be reported in the final report.

\section{Learn GPU programming}
\label{spec:ch:activities:learn-gpu-programming}

This step will launch the project and it can be without any knowledge of the project Celeritas, the objective is to learn the fundamentals of GPU programming.


\subsection{Learn CUDA}
\label{spec:ch:activities:learn-gpu-programming:learn-cuda}

First of all, the programming oriented \acrshort{gpu} and the language \acrshort{cuda} are totally new, there is no course about that in the Bachelor program.

\acrlong{lbl} provides a course about CUDA~\cite{cuda-training} and some exercises~\cite{cuda-series} to learn the language.


\subsection{Write cheat sheet}
\label{spec:spec:ch:activities:learn-gpu-programming:write-cheat-sheet}

For each course, a cheat sheet will be produced to summarize the important notions and provide quick access to the information during the realization of the project.
Not every course will produce a new one, an old cheat sheet can be improved.


\section{Understand the project}
\label{spec:ch:activities:understand-the-project}

The first step to do in the laboratory is to dive into the project to understand them.
The discussion with the team and the customer will be important to catch the important points.


\subsection{Compile the code}
\label{spec:ch:activities:understand-the-project:compile-the-code}

Celeritas is made to be run on \acrshort{hpc} and it comes with a new tool to build it.
This is important to understand the basis of Module~\cite{Module} and Spack~\cite{Spack} to be able to compile the code.
This task is the first one to do in the laboratory.


\subsection{Launch job on perlmutter}
\label{spec:ch:activities:understand-the-project:launch-job-on-perlmutter}
Perlmutter~\cite{Perlmutter} is the new \acrshort{hpc} of \acrshort{nersc}.
To launch a job on this kind of machine, it's not a simple command as on a personal computer, this requires a script that includes some parameters and the command to launch the code.
This step must be done to have an example that can be adapted to Celeritas.


\subsection{Profile the code}
\label{specspec:ch:activities:understand-the-project:profile-the-code}

In order to measure the improvement, the code must be profiled on Perlmutter to know the performance's reference.
This step includes running the application with the relevant input and having some basic knowledge of the Nvidia profiling tool.
It is very important to know the limit of the code and announce some objectives to reach before starting coding.
To close this second goal of the project, a profile must be recorded and analyzed.


\subsection{Background the project}
\label{spec:ch:activities:understand-the-project:background-the-project}

Aside from the other task, this one will be done to understand the background of the project.
It is not mandatory to drive this Bachelor thesis to success but it will help to understand what the other members of the team are doing and to provide a better view of the project in the final report.



\section{Improve the performance}
\label{spec:ch:activities:improve-the-performance}

This is the goal of this Bachelor's thesis and to do that, several steps must be done to reach the objective.


\subsection{Understand the code}
\label{spec:ch:activities:improve-the-performance:understand-the-code}

First of all, it's important to understand how the code is working in order to not reinvent the wheel with a function that already exists.
The objective is also to follow the guideline of the project to make easier understanding and maintainability for the team.


\subsection{Understand Runge-Kutta}
\label{spec:ch:activities:improve-the-performance:understand-runge-kutta}

As the optimization is on the Runge-Kutta method~\cite{Runge-Kutta-methods}, it's important to understand how it is working and how it's implemented in the code.
Some analysis must be done to know where it's possible to improve the performance and how.


\subsection{Implement the optimization}
\label{spec:ch:activities:improve-the-performance:implement-the-optimization}

Once all the analysis is done, it's time to implement a new version of the Runge-Kutta that uses all the advantages of the \acrshort{gpu}.
This step will be done in several iterations to be sure that the code is working and that the performance is improved.

Some sub-tasks will appear under this one but they will be defined during the project.

