\chapter{Activities}
\label{spec:ch:activities}

Activities are all the tasks that need to be done to complete the project. These tasks
come directly from the goals. The planning is based on the activities listed above.

All the tasks aren't mentioning the documentation, but it's implicit that all the tasks will be reported in the final report.

\section{Learn GPU Programming}
\label{spec:ch:activities:learn-gpu-programming}

This step will launch the project and it can be without any knowledge of the project Celeritas, the objective is to learn the fundamental of the GPU programming.


\subsection{Learn CUDA}
\label{spec:ch:activities:learn-gpu-programming:learn-cuda}

First of all, the programming oriented \acrshort{gpu} and the language \acrshort{cuda} are totally new, there no class about that in the bachelor program.

\acrlong{lbl} provide a course about CUDA~\cite{cuda-training} and some exercices~\cite{cuda-series} to learn the language. 


\subsection{Write cheat sheet}
\label{spec:spec:ch:activities:learn-gpu-programming:write-cheat-sheet}

For each course, a cheat sheet will be produced to summarize the important notions and provide a quick access to the information during the realization of the project.
Not every course will produce a new one, a cheat sheet can be improved.


\section{Understand the project}
\label{spec:ch:activities:understand-the-project}
The first step to do in the laboratory is to understand the project and the code.


\subsection{Compile the code}
\label{spec:ch:activities:understand-the-project:compile-the-code}
Celeritas is made to be run on \acrshort{hpc} and it come with new tool to build it.
The first step is to understand the basis of Module~\cite{Module} and Spack~\cite{Spack} to be able to compile the code.
This task is the first one to do in the laboratory.


\subsection{Launch Job on Perlmutter}
\label{spec:ch:activities:understand-the-project:launch-job-on-perlmutter}
Perlmutter~\cite{Perlmutter} is the new \acrshort{hpc} of \acrshort{nersc}.
To launch a job on this kind of machine, it's not a simple command as on a personal computer.
This require a script that include some parameters and the command to launch the code.
This step must be done before launching the code.


\subsection{Profile the code}
\label{specspec:ch:activities:understand-the-project:profile-the-code}
To basis performance, in order to measure the improvement, the code must be profiled on Perlmutter.
This step include launching the project and some basis knowledge of the Nvidia profiling tool.
It is very important to know the limit of the code and announce some objective to reach before starting coding.
When a profile is recorded and analyzed, this will close the second goal of the project.


\subsection{Background the project}
\label{spec:ch:activities:understand-the-project:background-the-project}
Aside the other task, this one will be done in order to understand the background of the project.
It is not mandatory to drive this bachelor thesis to the success but it will help to understand what the other members of the team are doing and to provide a better view of the project in the final report.



\section{Improve the performance}
\label{spec:ch:activities:improve-the-performance}
This is the goal of this bachelor's thesis and to do that, several steps must be done to reach the objective.


\subsection{Understand the code}
\label{spec:ch:activities:improve-the-performance:understand-the-code}
First of all, it's important to understand how the code is working to not reinvent the wheel with function that already exist.
The objective is also to follow the guideline of the project to make easier the understanding and the maintainability.


\subsection{Understand Runge-Kutta}
\label{spec:ch:activities:improve-the-performance:understand-runge-kutta}
As the optimization is on the Runge-Kutta method, it's important to understand how it's working and how it's implemented in the code.
Some analysis must be done to know where it's possible to improve the performance and how.


\subsection{Implement the optimization}
\label{spec:ch:activities:improve-the-performance:implement-the-optimization}
As all the analysis are done, it's time to implement a new version of the Runge-Kutta that use all the advantages of the \acrshort{gpu}.
This step will be done in several iteration to be sure that the code is working and that the performance are improved.


