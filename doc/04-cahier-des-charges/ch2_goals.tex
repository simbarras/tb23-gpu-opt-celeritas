\chapter{Goals}
\label{ch:goals}

The list of goals is the different deliverables of the project and the main ones will be found
in the list of milestones.
The goals are described with the \acrfull{smart} model~\cite{smart}.

At the end of the project, the \acrshort{mvp} will be able to control and read the information of the microdispenser.
It will also provide an interface to manage it.
The code is destined to be changed and improved in the future, so it will be written in a way that it can be easily modified.

\section{Mandatory requirements}
\label{sec:mandatory-requirements}

These requirements must be delivered at the end of the project.

\subsection{Control the $\mu$BoMa}
\label{subsec:control}
To specify this requirement, it is necessary to know that a microcontroller will be used to control the microdispenser.
This step is important because it's a blocking point for the project.

At the end of this goal, the user must be able to power on and off the microdispenser to move the vane and create a void or a pressure.
For now, the sequence to deliver a known volume of liquid isn't important.

This goal will launch the project and the next milestones will be based on it.
The main part of the analysis will be done in this goal and all the decisions taken will influence the future of the project.

To be finished quickly, this step will take all the efforts of the first week.
The due date is the end of the first milestones of the project.

\subsection{Read the information of the microdispenser}\label{subsec:read}
This goal is the second step of the project and the last of the critical part.
To complete this goal, the microcontroller will be able to read the pressure and the bubble detector of the microdispenser.
This information would allow us to finish the deliverable sequence to dispense a wanted volume of liquid.
This step isn't including the control of the volume by the user.

\subsection{Provide an interface to manage the microdispenser}\label{subsec:goals-interface}
The last requirement of the project will be done feature by feature.
The final state is to have one or many application interfaces that can be easily integrated in a production line and in a laboratory environment.
Those interface must be able to control a dispense and they will be on different supports and different protocols.
If the time allows it, they can be a user application like a web page or a mobile application.
The laboratory environment will be prioritized but it's important to keep in mind that the production line will be the final goal and this service must be over ethernet and the protocol \acrfull{opc}~\cite{opc}.



\section{Optional requirements}\label{sec:goals:optional}
These requirements are not mandatory but they will be done if the time allows it.

\subsection{Follow the standard SiLA}\label{subsec:sila}
The standard SiLA~\cite{sila} is a standard to control laboratory equipment.
It allows you to have a common protocol interface and to be discovered on the network.

\subsection{Follow the standard \acrlong{mtp}}\label{subsec:mtp}
Like SiLA is also a standard in the pharmaceutical industry.
The difference is that it's a standard to control the production line.

\subsection{More than one interface to manage the microdispenser}\label{subsec:moreinterface}
The goal is to provide more than one application interface to manage the microdispenser in multiple environments.
The first one will be the laboratory environment and the second one will be the production line with the \acrshort{opc}~\cite{opc}.
This protocol will use ethernet to communicate.

\subsection{Application to control the dispenser in a laboratory environment}\label{subsec:mobile}
To use the microdispenser in a laboratory environment, it's important to have a client application to control it.
The application can be a mobile, a desktop or a web application.
