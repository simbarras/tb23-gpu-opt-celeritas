\chapter{Goals}
\label{spec:ch:goals}

The list of goals is the different deliverables of the project and the main ones will be found in hte list of milestones.
The goals are described in the following sections using the \acrshort{smart} method.

At the end of the project, the profiling must show an improvement of the performance specialy in the part of the Runge-Kutta method~\cite{Runge-Kutta-methods}.
To mesure it, the profiling must be done before and after each step of the project.


\section{Mandatory requirements}
\label{spec:ch:goals:mandatory-requirements}

These requirements must be delivered at the end of the project.


\subsection{Learn GPU Programming}
\label{spec:ch:goals:mandatory-requirements:learn-gpu-programming}

Before starting to work on the project, some things need to be learned.
The goal of this step is to learn a new way of programming.
To conclude this goal, no code will be produced except for exercises, but the important notions of programming with CUDA will be synthesized using cheat sheets.
To take advantage of the delay between the begin of the bachelor thesis and the begin in the laboratory, this step will be done during this time.


\subsection{Understand the project}
\label{spec:ch:goals:mandatory-requirements:understand-the-project}

To be able to improve the performance of the code, the first step is understand the project.
It's always better to understand the background of the project.
Why it is needed, who will use it and which paradigm and tools are used.
To measure the performance gained, it necessary to know where the project is at each step.
To take a snapshot of the performance, a profile can be run.
This includa that we can compile and launch the project.
This step will be done at the begining of the project on place.

To compile the code, the tool Module~\cite{Module} and Spack~\cite{Spack} are used.
These tool don't need to be fully understood but the basic usage must be known to debug the compilation.

To launch the code, the profiler tool of Nvidia must be used and the script to run a job on Perlmutter~\cite{Perlmutter} must be known.
The knowledge gained in the previous step must be used to test the limit of the project.


\subsection{Improve the performance}
\label{spec:ch:goals:mandatory-requirements:improve-the-performance}

The main goal of the project is to improve the performance of the Runge-Kutta method~\cite{Runge-Kutta-methods}.
This last mandatory requirement is the aim of the thesis and the most important part of the project.
It will require the knowledge gained in the first two step to improve the performance.
To conclude this step, the profiling must show an improvement of the general performance and specialy in the part of the Runge-Kutta method~\cite{Runge-Kutta-methods}.
This step will be done after the first two steps and it will occupate the whole time left.


\section{Optional requirements}
\label{spec:ch:goals:optional-requirements}

These requirements are not mandatory but they are a good addition to the project.

\subsection{Another performance improvement}
\label{spec:ch:goals:optional-requirements:another-performance-improvement}

If the performance of the Runge-Kutta method~\cite{Runge-Kutta-methods} is improved, another part of the code can be improved.
This will be done only if the first part is done and if there is enough time to do it.
All the knowledge gained in the first two steps will be used the same way as the main goal.
To know which part of the code can be improved, the profiling must be done again.
To clsoe this step, the final profiling must show an improvement of the performance like the previous improvement.
This step can be done multiple times if there is enough time.

