\chapter{Goals}
\label{spec:ch:goals}

The list of goals enumerates the different deliverables of the project and the main ones will be found in the list of milestones.
The goals are described in the following sections using the \acrshort{smart} method.

At the end of the project, profiling should not necessarily show a performance improvement.
This is because we don't yet know whether thread synchronization will be more time-consuming than the original version.
In addition, the kernel launch in Celeritas may have to be changed, and this may take up a considerable amount of optimization time.
It is sufficient to have a proof of concept that demonstrates that the enhancement is effective and deserves to be integrated.
To measure it, the profiling must be done before and after each step of the project.

\section{Mandatory requirements}
\label{spec:ch:goals:mandatory-requirements}

These requirements must be delivered at the end of the project.


\subsection{Learn GPU Programming}
\label{spec:ch:goals:mandatory-requirements:learn-gpu-programming}

Before starting to work on the project, some things need to be learned and the goal here is to learn a new way of programming.
To conclude this goal, no code will be produced except for exercises, but the important notions of \acrshort{gpu} programming with CUDA will be synthesized using cheat sheets.
To take advantage of the delay between the beginning of the bachelor thesis and the beginning in the laboratory, this step will be done during this time.


\subsection{Understand the Project}
\label{spec:ch:goals:mandatory-requirements:understand-the-project}

To be able to improve the performance of the code, the first step is to understand the project and it's always better to understand the background: why it is needed, who will use it and which paradigm and tools are used.

To measure the performance gained, it is necessary to know where the project is at each step.
To take a snapshot of the performance, a profiler can be run and this includes that we can compile and launch the project.
This step will be done at the beginning of the project on-site.


\subsection{Improve the Performance}
\label{spec:ch:goals:mandatory-requirements:improve-the-performance}

The main goal of the project is to improve the performance of the implementation of dormand Prince method~\cite{princeDormand}.
This last mandatory requirement is the core of the thesis and the most important part of the project and it will require the knowledge gained in the first two steps to improve the performance.

To conclude this step, the code must compile, pass the unit test and a profiler must be run.
The profile shows the difference between the two implementations of the DormandPrince method.
As said in the general goal of the project, this task can be inefficient or too long to be finished, the goal is to have something that proves this modification must be integrated or not in the project.
This step will be done after the first two steps and it will take the whole time left.


\section{Optional requirements}
\label{spec:ch:goals:optional-requirements}

These requirements are not mandatory but they could be a good addition to the project.

\subsection{General architecture}
\label{spec:ch:goals:optional-requirements:general-optimization}

The purpose of Celeritas is to be run on all kinds of \acrshort{gpu} and even on machines with just have a \acrshort{cpu}.
During the optimization, the improvement will be checked on the Perlmutter~\cite{Perlmutter} which uses Nvidia A100 with the architecture Ampere~\cite{ampere} and some improvement can be only effective to this kind of \acrshort{gpu}.
This goal is here to check if the improvement has a positive effect on other architectures and if it is not, to find a way to do that.
To begin this step, the main goal needs to be finished.

\subsection{Another Performance Improvement}
\label{spec:ch:goals:optional-requirements:another-performance-improvement}

If the performance of the Runge-Kutta method~\cite{Runge-Kutta-methods} is improved, another optimization can be done.
This part goal will be discussed further in the project with the supervisors and the customers and it will be managed like the last mandatory goal.
This step can be done multiple times if there is enough time.

