\chapter{Context}
\label{chap:contexte}

This semester project (PS6) is a collaboration with the company \acrfull{dec}.
The goal is to realize a connected dispenser that allows micro-dosing for the production of drugs.


\section{Drug production}
\label{sec:context:production}

Today, drug production faces many challenges and she is becoming more and more complex and the demand is increasing.
We have seen at COVID-19 that the industry has had difficulty in meeting the demand.
To avoid this happening again, it's important to find solutions to increase production agility.

Currently, the production of drugs is done in huge quantities and the goal of this project is to provide a solution for small batch production and laboratory testing.


\section{DEC Group}
\label{sec:context:dec}

\acrfull{dec} is a Swiss-based multinational company that has been specializing in pharmaceutical, chemical and cosmetic production equipment for over 35 years.
This company is at the origin of the creation of the $\mu$BoMa.
This is a microdispenser that answers the demand for small batch production.
The figure \ref{fig:dec} shows the prototype of the microdispenser.

\begin{figure}[ht]
    \centering
    \includegraphics[width=0.4\textwidth]{img/logo.png}
    \caption{$\mu$BoMa illustration}
    \label{fig:dec}
\end{figure}

\acrshort{dec} has already produced the prototype $\mu$BoMa.
However, for the moment it's not autonomous and must be connected to an industrial computer using several cables to function.


\section{The need}
\label{sec:context:need}

The goal of this project is to realize the connected part of the $\mu$BoMa.
We want to be able to use the microdispenser in a production line that can manage up to three dispensers at the same time and in a laboratory environment where a smartphone or other personal device can control the dosage.
For this, it will be equipped with a microcontroller that will have a module to network with other devices.
We can imagine wifi and Bluetooth but it remains to be defined.
